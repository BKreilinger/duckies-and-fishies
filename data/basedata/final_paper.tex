\documentclass[11pt]{article}
\usepackage[utf8]{inputenc}
\usepackage{graphicx}
\usepackage{csvsimple}
\usepackage{fancyhdr}
\usepackage{geometry}
\usepackage{subcaption}

\geometry{
  a4paper,
  left=20mm,
  right=20mm,
  headheight=10cm,
  top=5.5cm,
  bottom=2cm,
  tmargin=30pt
}

\pagestyle{fancy}
\fancyhead{}
\fancyfoot{}
\fancyfoot[R]{\thepage}
\renewcommand{\headrulewidth}{0pt}

\newenvironment{heading}{\large \centering}\par
\newenvironment{textbody}{\normalsize}\par

\begin{document}

\begin{heading}
\textbf{Duckies and Fishies reproduction package}
\end{heading}

\vspace{0.5cm}

\begin{heading}
    Luca Escher \\
    Benjamin Kreilinger
\end{heading}

\vspace{0.5cm}

\section{Design choices}

\begin{textbody}
While reproducing the project “duckies and fishies” \footnote{Micheal Milton, Head First Data Analysis, 75-109} we were faced with several design and decision problems that had to be solved under the aspect of reproducibility.

In order to provide a reproducible environment which contains all necessary dependencies and packages, we have decided to go with docker, more specifically, docker-compose. This allows the user to build and start the environment using a single command, with all environment variables being set in a file. This also allows for future expandability and easy adjustments. \\

Next we had to opt for our main programming language. The two most fitting languages were Python and R. Both tools are very capable for this task, but because Python is the most spread programming language today, we chose python. \footnote{PYPL PopularitY of Programming Language: https://pypl.github.io/PYPL.html} \\
\end{textbody}

\begin{textbody}
In our next step we reproduced the graphs of the given paper. (reference) In our project we have chosen to use Plotly, due to easy readability, good documentation, the fact that the produced graphs are bitwise identical and you can easily customize the graphs for individual needs. \\


In order to provide all necessary data in the project and to avoid licensed software we decided to convert the original data, which was available as an Excel file, into a csv file and to provide only these csv files. This way the data remains easily usable but is much more independent.

Pandas is very efficient with small amounts of data (usually between 100 MB and 1 GB). With the help of this library it is possible to read and process csv files (even in their natural state) very easily and quickly, furthermore it is highly compatible with the Plotly library we have chosen for our project. \\

For our final documentation, which you are currently reading, we had to generate a PDF with all the special content generated by the python code. Therefore LaTex was the most suited option due to its customizability and consistent format. Even though the generated PDFs are not bitwise identical, it doesn’t matter since it is just a documentation. \\

While converting the data to a CSV file we noticed that the months were not clearly presented in the base data. \footnote{Historical sales data: https://resources.oreilly.com/examples/9780596153946/-/blob/master/historical\_sales\_data.xls} For example, the months of January, July and June were all represented by a "J". Since this may cause confusion when viewing the data rows individually, we converted the months to ['Jan', 'Feb', 'Mar', 'Apr', 'May', 'Jun', 'Jul', 'Aug', 'Sep', 'Oct', 'Nov', 'Dec'] when generating the "historical sales graph". We have not made this rewriting process on the base data itself, so that the user himself has the possibility to decide how he wants to deal with the data in general and view the original data without any manipulations. In this way we enable a neutral reproduction of our project with the exact data, which we used aswell. \\

In the paper, a so called "solver" is used to find the optimal profit at given conditions. This solver is part of the proprietary software Excel and therefore not suitable for reproduction. \footnote{Basedata: https://resources.oreilly.com/examples/9780596153946/-/blob/master/bathing\_friends\_unlimited.xls} For this reason, we set up a linear program, which can also give us the optimal result of an objective function. Python offers many different libraries to find the solution of a linear program. Especially wide spread are SciPy, PuLP and ortools.

We have decided to use PuLP since it has a more convenient linear programming API than SciPy. You don’t have to mathematically modify your problem or use vectors and matrices. Since the code is easily readable, it also offers the possibility to be adjusted and expanded. \\

We want to share our project freely with others so they can reproduce and expand our code. Therefore we have used the MIT License in order to promote our own names, hence we have to be mentioned in their project if they use our code, but we won’t be chargeable or legally accountable in any way. \\

Since the contributor Luca Escher used an M1 MacBook Air, problems with the arm architecture occured. Therefore we have dealt with this topic and finally set up a second docker container which is capable of running the linear program (/scripts/solver.py) which didn’t work in the default docker container.
\end{textbody}

\section{Reproduction process}

\begin{figure}[h]
\centering
    \begin{subfigure}[t]{0.5\textwidth}
    \centering
        \includegraphics[scale=0.35]{data/images/product_mix.png}
        \caption{Product mix 1 with constraints}
        \label{fig:product_mix_with_restrictions}
    \end{subfigure}%
    ~
    \begin{subfigure}[t]{0.5\textwidth}
    \centering
        \includegraphics[scale=0.35]{data/images/hist_sales_graph.png}
        \caption{Historical sales}
        \label{fig:hist_sales}
\end{subfigure}

\end{figure}

\vspace{1cm}

\begin{textbody}
    First we generated the product mix graph which contains product mix 1 and its constraints and therefore presents the valid scope of the objective function.
The python script \\ \emph{(/scripts/generate\_product\_mix\_graph.py)} generates this Figure
1a.
Next, we analyzed and graphically represented (Figure 1b) the sales data from Jan-2006 to Dec-2008 \\ \emph{(/scripts/reproduce\_historical\_sales\_data\_diagram.py)}.
\end{textbody}

\begin{table}[h]
    \centering
    \begin{tabular}{|l|l|l|}
        \hline
        \bfseries Quantity ducks & \bfseries Quantity fish & \bfseries Profit
        \csvreader[head to column names]{data/csv_result_documents/bathing_firends_res1.csv}{}
        {\\\hline\ \QuantityDucks & \QuantityFish & \Profit} \\
        \hline
    \end{tabular}
    \caption{Solver without estimated demand}
    \label{table:solver_without_estimated_demand}
\end{table}

\begin{textbody}

Lastly, we run the linear program implemented in \emph{/scripts/solver.py}. In the beginning we ignore the estimated demand which the paper assumes from the historical sales and solve the linear equation. The result is shown in Table 1. Afterwards the assumptions of the paper regarding the estimated demands are taken into account. Now the linear equation is solved again, the results can be found in Table 2.
\end{textbody}

\begin{table}[h]
    \centering
    \begin{tabular}{|l|l|l|}
        \hline
        \bfseries Quantity ducks & \bfseries Quantity fish & \bfseries Profit \\
        \hline
        \csvreader[head to column names]{data/csv_result_documents/bathing_firends_res2_with_estimated_demands.csv}{}
        {\QuantityDucks & \QuantityFish & \Profit} \\
        \hline
    \end{tabular}
    \caption{Solver with estimated demand}
    \label{table:solver_with_estimated_demand}
\end{table}
\begin{textbody}
Our project is considered a success, if it it possible to rerun the code n times and have a bitwise identical output. After many test runs, the graphs and csv files were always bitwise identical. The PDF is not bitwise identical, but the content stays the same, which is why we consider it a success.
\end{textbody}
\end{document}